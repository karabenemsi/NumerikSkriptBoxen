\documentclass[13pt]{scrreprt}
\usepackage[ngerman]{babel}
\usepackage{amsmath}
\usepackage{amsfonts}
\usepackage{amssymb}
\usepackage[utf8x]{inputenc}
\usepackage[T1]{fontenc}
\usepackage{geometry}

% Title Page
\title{Numerik Boxen}
\author{}


\newenvironment{ebox}[1]{\textbf{#1}\\}{}

\begin{document}
\maketitle

\begin{abstract}
\end{abstract}


\chapter{1}

11
\begin{equation*}
v + 134 = 16
\end{equation*}
	
12
\begin{equation*}
f_{rd}(r):= \delta_{r}:=rd(r)-r
\end{equation*}

13
\begin{equation*}
f_{rel}:=\epsilon_{r}:= \frac{\delta_{r}}{r}=\frac{rd(r) -r}{r} \quad\text{ für }\quad r \ne 0
\end{equation*}



16
\begin{equation*}
|\delta_{r}| \leqslant 2^{e-t} \qquad \text{und} \qquad  |\epsilon_{r}| \leqslant 2^{-t}
\end{equation*}

17
\begin{equation*}
\epsilon = 2^{-t}
\end{equation*}

18

...die entsprechenden Maschinenoperationen, d.h. die Anwendung der jeweiligen Operation auf Zahlen im normierten Gleitpunkzahlenformat nach Def. 1.2

19

I) Verknüpfung der Maschinenzahlen mit höherer (ausreichend hoher) Genauigkeit\\
II) Runden des Ergebnisses auf eine Maschinenzahl

\begin{ebox}{20}
	\begin{align*}
	r \circ _{M}s := rd(r \circ s)
	\end{align*}
\end{ebox}

\begin{ebox}{21}
	\begin{align*}
	r + _{M} s = (1,11)_2 \cdot 2^0 +_{M} (1,10)_2 \cdot 2^2 &= (111)_2 \cdot 2^{-2} + (1,10)_2 \cdot 2^{-2}\\
	&= (1000,10)_2 \cdot 2^{-2}\\
	&= (100010)_2 \cdot 2^1\\
	&= (1,00)_2 \cdot 2^1 = (2)_{10}
	\end{align*}
	wohingegen das exakte Ergebnis $ r+s = \frac74 + \frac38= \frac{17}{8}$ ist.\\
	\begin{align*}
	\text{Der absolute Rundengsfehler ist also}\\
	f_{rd} = (r+_M s) - (r + s) = 2 -\frac{17}{8} = -\frac{1}{8}\\
	\text{und der relative Rundungsfehler ist}\\
	|f_{rel}| = |\frac{(1 +_M s)-(r+s)}{r+s}| = \frac{\frac18}{\frac{17}{8}} = \frac{1}{17} = 0,0588... \approx 5,88\%\\
	\text{DAs ist relativ gut, denn der maximale Rundugnfehler bei einer 3-stelligen Mantisse ist} \quad \epsilon =   2^{-3} = 12,5\%
	\end{align*}
\end{ebox}


\begin{ebox}{23}
	\begin{align*}
	r \circ_M s = rd(r \circ s) = (r\circ s)\cdot(1 + \epsilon_0)
	\end{align*}
	wobei der realtie Fehler $\epsilon_0$ der Gleitpunktoperation wegen Def. 1.10 stets durh die Maschinengenauigkeit beschränkt ist,
	\begin{align*}
	\epsilon_0 := \frac{(r\circ_M s)-(r \circ s)}{r \circ s} = \frac{rd(r \circ s) - (r \circ s)}{r \circ s} \leqslant \epsilon
	\end{align*}
\end{ebox}

\begin{ebox}{24}
	\begin{align*}
		\tilde{y} =& \tilde{x} +_M c \\
		&= (\tilde{x} + c) \cdot (1+ \epsilon_2)\\
		&= ((a +_M b) + c) \cdot (1+ \epsilon_2)\\
		&= ((a + b) \cdot (1+ \epsilon_1) + c) \cdot (1+ \epsilon_2)\\
		&= a + b + c + (a+b)\cdot\epsilon_1 + (a + b +c)\cdot\epsilon_2 + (a+b)\cdot\epsilon_1\epsilon_2
	\end{align*}
\end{ebox}

\begin{ebox}{25}
	\begin{align*}
	\tilde{y} \overset{\cdot}{=}  a + b + c + (a+b)\cdot\epsilon_1 + (a + b +c)\cdot\epsilon_2
	\end{align*}
\end{ebox}

\begin{ebox}{26}
	\begin{align*}
	f_{rel}(y) = \frac{\tilde{y} - y}{y} &= \frac{(a+b)\cdot\epsilon_1 + (a + b +c)\cdot\epsilon_2}{a + b + c}\\
	&= \frac{a+b}{a+b+c}\cdot\epsilon_1 + \epsilon_2
	\end{align*}
\end{ebox}

\begin{ebox}{27}
	\begin{align}
	|f_{rel}(y)| &= |\frac{a+b}{a+b+c}\cdot\epsilon_1 + \epsilon_2|\\ &\overset{(D.U.G)}{\leqslant} |\frac{a+b}{a+b+c}|\cdot|\epsilon_1| + |\epsilon_2|
	\end{align}
\end{ebox}

\begin{ebox}{28}
	\begin{align*}
	|f_{rel}(y)| \leqslant (1+ | \frac{a+b}{a+b+c}|) \cdot \epsilon = (a+ \frac{1}{|1 + \frac{c}{a+b}|}) \cdot \epsilon
	\end{align*}
\end{ebox}

\begin{ebox}{29}
	\begin{align*}
	c \approx -(a+b)
	\end{align*}
	ist. Denn für $c \rightarrow -(a+b)$ ist $ \frac{|a+b|}{|a+b+c|} \rightarrow \infty$ und damit wird auch die obere Schranke von $|f_{rel}(y)|$ beliebig (unendlich) groß. In diesem Fall spricht man von \textbf{Auslöschung}
\end{ebox}


\end{document}          
