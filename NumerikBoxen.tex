\documentclass[13pt]{scrreprt}
\usepackage[ngerman]{babel}
\usepackage{amsmath}
\usepackage{amsfonts}
\usepackage{amssymb}
\usepackage[utf8x]{inputenc}
\usepackage[T1]{fontenc}
\usepackage{geometry}
\usepackage{environ}
\usepackage{enumitem}
\usepackage[many]{tcolorbox}


% Title Page
\title{Numerik Boxen}
\author{Florian Lubitz \& Steffen Hecht}

\newcounter{BoxCounter}
\setcounter{BoxCounter}{0}
\NewEnviron{tbox}{%
	\stepcounter{BoxCounter}
	\textbf{\arabic{BoxCounter}}
	\BODY
}
\NewEnviron{abox}{%
	\stepcounter{BoxCounter}
	\textbf{\arabic{BoxCounter}}
		\begin{align*}
		\BODY
		\end{align*}
}

\newtcolorbox{dbox}[1][]{%
	enhanced,frame hidden,interior hidden,
	arc=0pt,outer arc=0pt,borderline={0.4pt}{0pt}{dashed},
	nobeforeafter, box align=center, before={\stepcounter{BoxCounter}\boxed{\arabic{BoxCounter}}\makebox[.5cm]{}}, after={\vspace{0.5cm}}
}

\makeatletter
\newcommand*{\rom}[1]{\expandafter\@slowromancap\romannumeral #1@}
\makeatother

\begin{document}
\maketitle

\chapter{1}

\begin{dbox}
\end{dbox}

\begin{dbox}
	Die \underline{Gleitpunktzahlendarstellung} zerlegt jede reelle Zahl $r \in \mathbb{R}$ in drei Bestandteile:\\
	\begin{enumerate}[label=\Roman*)]
		\item ein Vorzeichen $ (-1)^v$ mit $v \in \{0, 1\}$
		\item eine Mantisse $m \in \mathbb{R}$ und
		\item einen Exponenten $e \in \mathbb{Z}$, sodass
	\end{enumerate}
	\begin{equation}
	r = (-1)^v \cdot m \cdot 2^e
	\end{equation}
\end{dbox}

\begin{dbox}
	\begin{equation}
	r = (-1)^v \cdot \sum_{i = 0}^{t - 1} r_i \cdot 2^{-i} \text{ mit } r_0 = 1 \text{ (sodass } 1 \le m \le 2\text{)}
	\end{equation}
\end{dbox}

\begin{dbox}
	\begin{align*}
	13,6 &= 2^3 + 5,6\\
	&= 2^3 + 2^2 + 1,6\\
	&= ...\\
	&= 2^3 + 2^2 + 2^0 + 2^{-1} + 2^{-4} + 0,0375\\
	&= ...
	\end{align*}
	Also können wir die Dezimalzahl $(13,6)_{10}$ als Binärzahl $(1101,1001...)_2$ darstellen.
\end{dbox}

\begin{dbox}
	\begin{align*}
	(13,6)_{10} &= (1101,1001...)_2\\
	&= (-1)^0 \cdot (1,1011001...)_2 \cdot 2^3
	\end{align*}
	also $v = 0$, $m = (1,1011001...)_2$ und $e = 3$
\end{dbox}

\begin{dbox}
	Eine reelle Zahl $r \in \mathbb{R}$ mit der Darstellung\\
	\begin{equation}
	r = (-1)^v \cdot \sum_{i = 0}^{t - 1} r_i \cdot 2^{-i} \cdot 2^e \text{ mit } r_0 = 0
	\end{equation}
	gehört der Menge der nicht-normierten oder subnormalen Gleitpunktzahlen $\mathbb{M}_s$ an.
\end{dbox}

\begin{dbox}
	Es gilt analog zu Definition 1.2: Das Vorzeichen $(-1)^v$ wird durch das Vorzeichen-Bit $v \in \{0, 1\}$, der Wert der Mantisse $m$ durch die Ziffern $r_i \in \{0, 1\}$, $i = 1,..., t - 1$ und der Wert des Exponenten durch eine ganze Zahl $e \in \mathbb{Z}$ mit $e = e_{min}$ festgelegt.
\end{dbox}

\begin{dbox}
	\begin{equation}
	0 = (-1)^v \cdot 0 \cdot 2^{e_{min}}
	\end{equation}
\end{dbox}

\setcounter{BoxCounter}{10}

\begin{abox}
v + 134 = 16
\end{abox}
	
\begin{abox}
f_{rd}(r):= \delta_{r}:=rd(r)-r
\end{abox}

\begin{abox}
f_{rel}:=\epsilon_{r}:= \frac{\delta_{r}}{r}=\frac{rd(r) -r}{r} \quad\text{ für }\quad r \ne 0
\end{abox}

\setcounter{BoxCounter}{15}
\begin{abox}
|\delta_{r}| \leqslant 2^{e-t} \qquad \text{und} \qquad  |\epsilon_{r}| \leqslant 2^{-t}
\end{abox}

\begin{abox}
\epsilon = 2^{-t}
\end{abox}

\begin{tbox}
...die entsprechenden Maschinenoperationen, d.h. die Anwendung der jeweiligen Operation auf Zahlen im normierten Gleitpunkzahlenformat nach Def. 1.2
\end{tbox}

\begin{tbox}
I) Verknüpfung der Maschinenzahlen mit höherer (ausreichend hoher) Genauigkeit\\
II) Runden des Ergebnisses auf eine Maschinenzahl
\end{tbox}

\begin{abox}{20}
	r \circ _{M}s := rd(r \circ s)
\end{abox}

\begin{tbox}{21}
	\begin{align*}
	r + _{M} s = (1,11)_2 \cdot 2^0 +_{M} (1,10)_2 \cdot 2^2 &= (111)_2 \cdot 2^{-2} + (1,10)_2 \cdot 2^{-2}\\
	&= (1000,10)_2 \cdot 2^{-2}\\
	&= (100010)_2 \cdot 2^1\\
	&= (1,00)_2 \cdot 2^1 = (2)_{10}
	\end{align*}
	wohingegen das exakte Ergebnis $ r+s = \frac74 + \frac38= \frac{17}{8}$ ist.\\
	\begin{align*}
	\text{Der absolute Rundengsfehler ist also}\\
	f_{rd} = (r+_M s) - (r + s) = 2 -\frac{17}{8} = -\frac{1}{8}\\
	\text{und der relative Rundungsfehler ist}\\
	|f_{rel}| = |\frac{(1 +_M s)-(r+s)}{r+s}| = \frac{\frac18}{\frac{17}{8}} = \frac{1}{17} = 0,0588... \approx 5,88\%\\
	\text{DAs ist relativ gut, denn der maximale Rundugnfehler bei einer 3-stelligen Mantisse ist} \quad \epsilon =   2^{-3} = 12,5\%
	\end{align*}
\end{tbox}


\begin{tbox}{23}
	\begin{align*}
	r \circ_M s = rd(r \circ s) = (r\circ s)\cdot(1 + \epsilon_0)
	\end{align*}
	wobei der realtie Fehler $\epsilon_0$ der Gleitpunktoperation wegen Def. 1.10 stets durh die Maschinengenauigkeit beschränkt ist,
	\begin{align*}
	\epsilon_0 := \frac{(r\circ_M s)-(r \circ s)}{r \circ s} = \frac{rd(r \circ s) - (r \circ s)}{r \circ s} \leqslant \epsilon
	\end{align*}
\end{tbox}

\begin{abox}{24}
		\tilde{y} =& \tilde{x} +_M c \\
		&= (\tilde{x} + c) \cdot (1+ \epsilon_2)\\
		&= ((a +_M b) + c) \cdot (1+ \epsilon_2)\\
		&= ((a + b) \cdot (1+ \epsilon_1) + c) \cdot (1+ \epsilon_2)\\
		&= a + b + c + (a+b)\cdot\epsilon_1 + (a + b +c)\cdot\epsilon_2 + (a+b)\cdot\epsilon_1\epsilon_2
\end{abox}

\begin{abox}{25}
	\tilde{y} \overset{\cdot}{=}  a + b + c + (a+b)\cdot\epsilon_1 + (a + b +c)\cdot\epsilon_2
\end{abox}

\begin{abox}{26}
	f_{rel}(y) = \frac{\tilde{y} - y}{y} &= \frac{(a+b)\cdot\epsilon_1 + (a + b +c)\cdot\epsilon_2}{a + b + c}\\
	&= \frac{a+b}{a+b+c}\cdot\epsilon_1 + \epsilon_2
\end{abox}

\begin{abox}{27}
	|f_{rel}(y)| &= |\frac{a+b}{a+b+c}\cdot\epsilon_1 + \epsilon_2|\\ &\overset{(D.U.G)}{\leqslant} |\frac{a+b}{a+b+c}|\cdot|\epsilon_1| + |\epsilon_2|
\end{abox}

\begin{abox}{28}
	|f_{rel}(y)| \leqslant (1+ | \frac{a+b}{a+b+c}|) \cdot \epsilon = (a+ \frac{1}{|1 + \frac{c}{a+b}|}) \cdot \epsilon
\end{abox}

\begin{tbox}{29}
	\begin{align*}
	c \approx -(a+b)
	\end{align*}
	ist. Denn für $c \rightarrow -(a+b)$ ist $ \frac{|a+b|}{|a+b+c|} \rightarrow \infty$ und damit wird auch die obere Schranke von $|f_{rel}(y)|$ beliebig (unendlich) groß. In diesem Fall spricht man von \textbf{Auslöschung}
\end{tbox}


\end{document}          
