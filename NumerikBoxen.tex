\documentclass[fontsize=13pt, parskip=half]{scrreprt}
\usepackage[ngerman]{babel}
\usepackage{amsmath}
\usepackage{amsfonts}
\usepackage{amssymb}
\usepackage[utf8x]{inputenc}
\usepackage[T1]{fontenc}
\usepackage{geometry}
\usepackage{environ}
\usepackage{enumitem}
\usepackage[many]{tcolorbox}
\usepackage{xcolor}


% Title Page
\title{Numerik Boxen}
\author{Florian Lubitz \& Steffen Hecht}

\newcounter{BoxCounter}
\setcounter{BoxCounter}{0}


\newtcolorbox{dbox}[1][]{%
	enhanced,frame hidden,interior hidden,
	arc=0pt,outer arc=0pt,borderline={0.4pt}{0pt}{dashed},
	nobeforeafter, box align=center, before={\stepcounter{BoxCounter}\boxed{\arabic{BoxCounter}}\makebox[.5cm]{}}, after={\vspace{0.5cm}}
}

\NewEnviron{tbox}{%
	\begin{dbox}
		\BODY
	\end{dbox}
}
\NewEnviron{abox}{%
	\begin{dbox}
		\begin{align*}
		\BODY
		\end{align*}
	\end{dbox}
}
\makeatletter
\newcommand\numbereq{%
	\ifmeasuring@
	\else
		\refstepcounter{equation}%
	\fi
	\tag{\theequation}%
}


\newcommand*{\rom}[1]{\expandafter\@slowromancap\romannumeral #1@}
\makeatother

\begin{document}

\maketitle

\chapter{1}

\begin{dbox}
\end{dbox}

\begin{tbox}
	Die \underline{Gleitpunktzahlendarstellung} zerlegt jede reelle Zahl $r \in \mathbb{R}$ in drei Bestandteile:\\
	\begin{enumerate}[label=\Roman*\right)]
		\item ein Vorzeichen $\left(-1\right)^v$ mit $v \in \{0, 1\}$
		\item eine Mantisse $m \in \mathbb{R}$ und
		\item einen Exponenten $e \in \mathbb{Z}$, sodass
	\end{enumerate}
	\begin{equation}
	r = \left(-1\right)^v \cdot m \cdot 2^e
	\end{equation}
\end{tbox}

\begin{tbox}
	\begin{equation}
	r = \left(-1\right)^v \cdot \sum_{i = 0}^{t - 1} r_i \cdot 2^{-i} \text{ mit } r_0 = 1 \text{ \left(sodass } 1 \le m \le 2\text{\right)}
	\end{equation}
\end{tbox}

\begin{tbox}
	\begin{align*}
	13,6 &= 2^3 + 5,6\\
	&= 2^3 + 2^2 + 1,6\\
	&= ...\\
	&= 2^3 + 2^2 + 2^0 + 2^{-1} + 2^{-4} + 0,0375\\
	&= ...
	\end{align*}
	Also können wir die Dezimalzahl $\left(13,6\right)_{10}$ als Binärzahl $\left(1101,1001...\right)_2$ darstellen.
\end{tbox}

\begin{tbox}
	\begin{align*}
	\left(13,6\right)_{10} &= \left(1101,1001...\right)_2\\
	&= \left(-1\right)^0 \cdot \left(1,1011001...\right)_2 \cdot 2^3
	\end{align*}
	also $v = 0$, $m = \left(1,1011001...\right)_2$ und $e = 3$
\end{tbox}

\begin{tbox}
	Eine reelle Zahl $r \in \mathbb{R}$ mit der Darstellung\\
	\begin{equation}
	r = \left(-1\right)^v \cdot \sum_{i = 0}^{t - 1} r_i \cdot 2^{-i} \cdot 2^e \text{ mit } r_0 = 0
	\end{equation}
	gehört der Menge der nicht-normierten oder subnormalen Gleitpunktzahlen $\mathbb{M}_s$ an.
\end{tbox}

\begin{tbox}
	Es gilt analog zu Definition 1.2: Das Vorzeichen $\left(-1\right)^v$ wird durch das Vorzeichen-Bit $v \in \{0, 1\}$, der Wert der Mantisse $m$ durch die Ziffern $r_i \in \{0, 1\}$, $i = 1,..., t - 1$ und der Wert des Exponenten durch eine ganze Zahl $e \in \mathbb{Z}$ mit $e = e_{min}$ festgelegt.
\end{tbox}

\begin{abox}
	0 = \left(-1\right)^v \cdot 0 \cdot 2^{e_{min}} \numbereq
\end{abox}

\setcounter{BoxCounter}{10}

\begin{abox}
v + 134 = 16
\end{abox}
	
\begin{abox}
f_{rd}\left(r\right):= \delta_{r}:=rd\left(r\right)-r
\end{abox}

\begin{abox}
f_{rel}:=\epsilon_{r}:= \frac{\delta_{r}}{r}=\frac{rd\left(r\right) -r}{r} \quad\text{ für }\quad r \ne 0
\end{abox}

\setcounter{BoxCounter}{15}
\begin{abox}
|\delta_{r}| \leqslant 2^{e-t} \qquad \text{und} \qquad  |\epsilon_{r}| \leqslant 2^{-t}
\end{abox}

\begin{abox}
\epsilon = 2^{-t}
\end{abox}

\begin{tbox}
...die entsprechenden Maschinenoperationen, d.h. die Anwendung der jeweiligen Operation auf Zahlen im normierten Gleitpunkzahlenformat nach Def. 1.2
\end{tbox}

\begin{tbox}
I\right) Verknüpfung der Maschinenzahlen mit höherer \left(ausreichend hoher\right) Genauigkeit\\
II\right) Runden des Ergebnisses auf eine Maschinenzahl
\end{tbox}

\begin{abox}
	r \circ _{\mathbb{M}}s := rd\left(r \circ s\right)
\end{abox}

\begin{tbox}
	\begin{align*}
	r + _{\mathbb{M}} s = \left(1,11\right)_2 \cdot 2^0 +_{\mathbb{M}} \left(1,10\right)_2 \cdot 2^2 &= \left(111\right)_2 \cdot 2^{-2} + \left(1,10\right)_2 \cdot 2^{-2}\\
	&= \left(1000,10\right)_2 \cdot 2^{-2}\\
	&= \left(100010\right)_2 \cdot 2^1\\
	&= \left(1,00\right)_2 \cdot 2^1 = \left(2\right)_{10}
	\end{align*}
	wohingegen das exakte Ergebnis $ r+s = \frac74 + \frac38 = \frac{17}{8}$ ist.\\
	Der absolute Rundengsfehler ist also
	\begin{align*}
	f_{rd} = \left(r+_{\mathbb{M}} s\right) - \left(r + s\right) = 2 -\frac{17}{8} = -\frac{1}{8}
	\end{align*}
	und der relative Rundungsfehler ist
	\begin{align*}
	|f_{rel}| &= |\frac{\left(1 +_{\mathbb{M}} s\right)-\left(r+s\right)}{r+s}|\\
	&= \frac{\frac18}{\frac{17}{8}}\\
	&= \frac{1}{17}\\
	&= 0,0588... \approx 5,88\%
\end{align*}
	Das ist relativ gut, denn der maximale Rundungsfehler bei einer 3-stelligen Mantisse ist
	$\epsilon =   2^{-3} = 12,5\%$.
\end{tbox}

\stepcounter{BoxCounter}

\begin{tbox}
	\begin{align*}
	r \circ_{\mathbb{M}} s = rd\left(r \circ s\right) = \left(r\circ s\right)\cdot\left(1 + \epsilon_0\right)
	\end{align*}
	wobei der relative Fehler $\epsilon_0$ der Gleitpunktoperation wegen Def. 1.10 stets durh die Maschinengenauigkeit beschränkt ist,
	\begin{align*}
	\epsilon_0 := \frac{\left(r\circ_{\mathbb{M}} s\right)-\left(r \circ s\right)}{r \circ s} = \frac{rd\left(r \circ s\right) - \left(r \circ s\right)}{r \circ s} \leqslant \epsilon
	\end{align*}
\end{tbox}

\begin{abox}
		\tilde{y} =& \tilde{x} +_{\mathbb{M}} c \\
		&= \left(\tilde{x} + c\right) \cdot \left(1+ \epsilon_2\right)\\
		&= \left(\left(a +_{\mathbb{M}} b\right) + c\right) \cdot \left(1+ \epsilon_2\right)\\
		&= \left(\left(a + b\right) \cdot \left(1+ \epsilon_1\right) + c\right) \cdot \left(1+ \epsilon_2\right)\\
		&= a + b + c + \left(a+b\right)\cdot\epsilon_1 + \left(a + b +c\right)\cdot\epsilon_2 + \left(a+b\right)\cdot\epsilon_1\epsilon_2
\end{abox}

\begin{abox}
	\tilde{y} \overset{\cdot}{=}  a + b + c + \left(a+b\right)\cdot\epsilon_1 + \left(a + b +c\right)\cdot\epsilon_2
\end{abox}

\begin{abox}
	f_{rel}\left(y\right) = \frac{\tilde{y} - y}{y} &= \frac{\left(a+b\right)\cdot\epsilon_1 + \left(a + b +c\right)\cdot\epsilon_2}{a + b + c}\\
	&= \frac{a+b}{a+b+c}\cdot\epsilon_1 + \epsilon_2
\end{abox}

\begin{abox}
	|f_{rel}\left(y\right)| &= |\frac{a+b}{a+b+c}\cdot\epsilon_1 + \epsilon_2|\\ &\overset{\left(D.U.G\right)}{\leqslant} |\frac{a+b}{a+b+c}|\cdot|\epsilon_1| + |\epsilon_2|
\end{abox}

\begin{abox}
	|f_{rel}\left(y\right)| \leqslant \left(1+ | \frac{a+b}{a+b+c}|\right) \cdot \epsilon = \left(a+ \frac{1}{|1 + \frac{c}{a+b}|}\right) \cdot \epsilon
\end{abox}

\begin{tbox}
	\begin{align*}
	c \approx -\left(a+b\right)
	\end{align*}
	ist. Denn für $c \rightarrow -\left(a+b\right)$ ist $ \frac{|a+b|}{|a+b+c|} \rightarrow \infty$ und damit wird auch die obere Schranke von $|f_{rel}\left(y\right)|$ beliebig \left(unendlich\right) groß. In diesem Fall spricht man von \textbf{Auslöschung}
\end{tbox}


30-36

\setcounter{BoxCounter}{36}
\setcounter{equation}{11}

\begin{tbox}
	\begin{align*}
	y + \delta_{y} = f\left(x + \delta_{x}\right) &= f\left(x\right) + f'\left(x\right)\delta_{x} + O\left(\delta_{x}^2\right)
	\end{align*}
	Unter Vernachlässigung der Terme höherer Ordnung gilt also:
	\begin{align*}
	\delta_{y} \overset{\cdot}{=} f'\left(x\right) \cdot \delta_{x}
	\end{align*}
	Die Verstärkung des relativen Rundungsfehlers $\epsilon_y$ des Resultats $y$ ergibt sich dann für $x,y \ne 0$:
	\begin{align*}
	\epsilon_y = f_{rel}\left(y\right) = \frac{\delta_{y}}{y} \overset{\cdot}{=} \frac{f'\left(x\right)\cdot \delta_{x}}{y} = \frac{x \dot f'\left(x\right)}{y} \cdot \frac{\delta_{x}}{x} = \frac{x \cdot f'\left(x\right)}{y} \cdot \epsilon_x \numbereq
	\end{align*}
	aus dem relativen Rundungsfehler $\epsilon_{x}$ der Eingabe $x$.
\end{tbox}

\begin{tbox}
	... aus dem Verstärkungsfaktor
	\begin{align*}
		K_f\left(x\right) := |\frac{x \cdot f'\left(x\right)}{f\left(x\right)}|
	\end{align*}
	zwschen den relativen Rundungsfehlern aus \left(1.12\right)
\end{tbox}

\chapter{Lineare Gleichungssysteme}

\begin{abox}
\end{abox}
\setcounter{BoxCounter}{39}

\begin{tbox}
\begin{align*}
10 x_1 + 7x_2 + 0x_3 &= 7\\
2,5x_2 + 5x_3 &= 2,5\\
6,2x_3 &= 6,2
\end{align*}
... oder in Matrixschreibweise:
\begin{align*}
\begin{pmatrix}
10 & -7 & 0\\
0 & 2,5 & 5\\
0 & 0 & 6,2
\end{pmatrix} \cdot \begin{pmatrix}
x_1 \\ x_2 \\ x_3
\end{pmatrix} = \begin{pmatrix}
7 \\ 2,5 \\ 6,2
\end{pmatrix}
\end{align*}
\end{tbox}

\begin{tbox}
	\begin{align*}
x_3 &= \frac{6,2}{6,2} = 1 \\
x_2 &= \frac{2,5-5x_3}{2,5} = \frac{2,4-5\cdot1}{2,5} = -1\\
x_1 &=  \frac{7-0x_2+7x_2}{10} = \frac{7-0-7}{10} = 0
\end{align*}
Somit resultiert der Lösungsvektor $\begin{pmatrix}
x_1\\x_2\\x_3
\end{pmatrix} = \begin{pmatrix}0\\-1\\1\end{pmatrix}$
\end{tbox}

\begin{tbox}
 \begin{enumerate}[label=\Roman*\right)]
 \item Berechnung $x_n = \frac{b_n}{a_nn}$
 \item Schrittweise Berechnung :\\
 	Für $i = n-1, n-2, ... , 3,2,1$ ist $x_i= \frac{b_i - \sum_{j=i+1}^{n}a_{ij} \cdot x_j}{a_{ii}}$
 \end{enumerate}
\end{tbox}

\begin{abox}
	\begin{pmatrix}
		a_{11}& a_{12} & a{1n} \\
a_{21}& a_{22} & a{2n} \\
\vdots \\
a_{n1} & a_{n2} & a_{nn}
	\end{pmatrix}
 \cdot \begin{pmatrix}
 	x_1 \\ x_2\\ \vdots \\ x_n \end{pmatrix}
  = \begin{pmatrix}
  	b_1 \\ b_2 \\ \vdots\\ b_n
  \end{pmatrix}
\end{abox}

\begin{abox}
	\begin{pmatrix}
		10 & -7 & 0\\
		-3 & 2 & 6\\
		5 & -1 & 5
	\end{pmatrix}
\cdot \begin{pmatrix}
	x_1 \\x_2 \\ x_3
\end{pmatrix} = \begin{pmatrix}
7 \\ 4 \\ 6
\end{pmatrix}
\end{abox}

\begin{abox}
	\begin{pmatrix}
	10 & -7 & 0\\
	0 & -0.1 & 6\\
	5 & -1 & 5
\end{pmatrix}
\cdot 
\begin{pmatrix}
	x_1 \\x_2 \\ x_3
\end{pmatrix} =
\begin{pmatrix}
7 \\ 6,1 \\ 6
\end{pmatrix}
\end{abox}

\begin{abox}
	\begin{pmatrix}
		10 & -7 & 0\\
		0 & -0.1 & 6\\
		0 & 2,5 & 5
	\end{pmatrix}
	\cdot 
	\begin{pmatrix}
		x_1 \\x_2 \\ x_3
	\end{pmatrix} =
	\begin{pmatrix}
		7 \\ 6,1 \\ 2,5
	\end{pmatrix}
\end{abox}

\begin{abox}
	\begin{pmatrix}
		10 & -7 & 0\\
		0 & -0.1 & 6\\
		0 & 0 & 155
	\end{pmatrix}
	\cdot 
	\begin{pmatrix}
		x_1 \\x_2 \\ x_3
	\end{pmatrix} =
	\begin{pmatrix}
		7 \\ 6,1 \\ 155
	\end{pmatrix}
\end{abox}

\begin{abox}
	x_3 &= \frac{155}{155} = 1\\
	x_2 &= \frac{6,1 - 6 \cdot 1}{-0,1} = -1\\
	x_1 &= \frac{7-\left(-7\right)\cdot\left(-1\right)-0\cdot1}{10} = 0
\end{abox}

\begin{abox}
	\begin{pmatrix}
		10 &-7 & 0\\
		0 & 0 & 6\\
		0 & 2,5 & 5
	\end{pmatrix} \cdot
\begin{pmatrix}
	x_1\\x_2\\x_3
\end{pmatrix} = \begin{pmatrix}
7 \\ 6,1 \\ 2,5
\end{pmatrix}
\end{abox}

\begin{abox}
	\begin{pmatrix}
		10 &-7 & 0\\
		0 & 2,5 & 5\\
				0 & -0,1 & 6
	\end{pmatrix} \cdot
	\begin{pmatrix}
		x_1\\x_2\\x_3
	\end{pmatrix} = \begin{pmatrix}
		7 \\ 2,5 \\ 6,1
	\end{pmatrix}
\end{abox}

\begin{abox}
	\begin{pmatrix}
		10 &-7 & 0\\
		0 & 2,5 & 5\\
		0 & 0 & 6,2
	\end{pmatrix} \cdot
	\begin{pmatrix}
		x_1\\x_2\\x_3
	\end{pmatrix} = \begin{pmatrix}
		7 \\ 2,5 \\ 6,2
	\end{pmatrix}
\end{abox}

\begin{tbox}
	Für eine $k \times n$ Matrix A und eine $n \times m$ Matrix B
	
	\begin{align*}
	A =
	\begin{pmatrix}
	a_{11} & \dots & a_{1n}\\
	\vdots & & \vdots \\
	a_{k1}  & \dots & a_{kn}
	\end{pmatrix}
	\text{ und } B = \begin{pmatrix}
		b_{11} & \dots & b_{1m}\\
		\vdots & & \vdots \\
		b_{n1}  & \dots & b_{bm}
	\end{pmatrix}
		\end{align*}
		
		ergibt sich das \underline{Matrixprodukt} $A \cdot B$ durch die $k \times m$ Matrix
		
		\begin{align*}
			A \cdot B = \begin{pmatrix}
			\sum_{j=1}^{n}a_{1j}\cdot b_{j1} & \dots & \sum_{j=1}^{n}a_{1j}\cdot b_{jm} \\
			\vdots & & \vdots \\
			\sum_{j=1}^{n}a_{kj}\cdot b_{j1} & \dots & \sum_{j=1}^{n}a_{kj}\cdot b_{jm}
			\end{pmatrix}
		\end{align*}
\end{tbox}

\begin{abox}
	\left(AB\right)_{rs} = \sum_{j=1}^{n} a_{rj}b_{js}
\end{abox}

\begin{abox}
	A \cdot \left(B \cdot C\right) &= \left(A \cdot B\right) \cdot C\\
	A\cdot \left(B + C\right) &= A \cdot B + A \cdot C
\end{abox}

% 55
\begin{abox}
	C = x^T \cdot y = \left(x_1, ... , x_n\right) \cdot \begin{pmatrix}
		y_1 \\ \vdots \\ y_n
	\end{pmatrix} = 
\sum_{j=1}^{n} x_j \cdot y_j \quad \in \mathbb{R}
\end{abox}

Missing: 56-61

\setcounter{BoxCounter}{61}

\begin{abox}
	A\cdot x = b \text{ mit } A=\begin{pmatrix}
		a_{11} & ... & a_{1n}\\
		\vdots & & \vdots \\
		a_{m1} & ... & a_{mn}
	\end{pmatrix}, x=\begin{pmatrix}
	x_1 \\ \vdots \\ x_n
\end{pmatrix}, b= \begin{pmatrix}
b_1 \\ \vdots \\ b_n
\end{pmatrix}\text{ für } \underline{m>n}
\end{abox}

\begin{tbox}
	\begin{align*}
		x = \arg\min_{x'} \left|\left|Ax' - b\right|\right|_{2}^2
	\end{align*}
	genügt. Man bezeichnet dieses Verfahren auch als Methode der kleinsten Quadrate.
\end{tbox}

\begin{abox}
	Ax-b = \begin{pmatrix}
		\sum_{j=1}^{n} a_{1j}x_{j}-b_1 \\
		\vdots \\
		\sum_{j=1}^{n} a_{mj}x_{j}-b_m \\
	\end{pmatrix}
\end{abox}

\begin{abox}
	f\left(x\right):= \left|\left|Ax-b\right|\right|_2^2 = \sum_{k=1}^{m}\left(\sum_{j=1}^{n} a_{kj} \cdot x_{j} - b_{k}\right)^2
\end{abox}

\begin{abox}
	0 \overset{!}{=} \dfrac{6f}{6x_i} = \sum_{K=1}^{m}2\cdot \left(a_{Kj}\cdot x_{j} - b_K\right) \cdot a_{Ki} \text{ für } i =1 ,...,n \\
	\Leftrightarrow \sum_{K=1}^{m}a_{Ki} \sum_{j=1}^{n}a_{Kj} \cdot x_{j} = \sum_{K=1}^{m}a_{Ki} \cdot b_{K} \text{ für } i =1 ,...,n \\
\end{abox}

\begin{abox}
	\left(A^TAx\right)_i \overset{!}{=} \left(A^Tb\right)_i \text{ für } i = 1, ..., n
\end{abox}
\begin{abox}
	A^TAx = A^Tb\\
	\text{Diese Gleichung wird als \underline{Normalgleichung} zu A und b bezeichnet.}
\end{abox}
\end{document}          
